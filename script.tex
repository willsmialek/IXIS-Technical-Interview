% Options for packages loaded elsewhere
\PassOptionsToPackage{unicode}{hyperref}
\PassOptionsToPackage{hyphens}{url}
%
\documentclass[
]{article}
\usepackage{amsmath,amssymb}
\usepackage{iftex}
\ifPDFTeX
  \usepackage[T1]{fontenc}
  \usepackage[utf8]{inputenc}
  \usepackage{textcomp} % provide euro and other symbols
\else % if luatex or xetex
  \usepackage{unicode-math} % this also loads fontspec
  \defaultfontfeatures{Scale=MatchLowercase}
  \defaultfontfeatures[\rmfamily]{Ligatures=TeX,Scale=1}
\fi
\usepackage{lmodern}
\ifPDFTeX\else
  % xetex/luatex font selection
\fi
% Use upquote if available, for straight quotes in verbatim environments
\IfFileExists{upquote.sty}{\usepackage{upquote}}{}
\IfFileExists{microtype.sty}{% use microtype if available
  \usepackage[]{microtype}
  \UseMicrotypeSet[protrusion]{basicmath} % disable protrusion for tt fonts
}{}
\makeatletter
\@ifundefined{KOMAClassName}{% if non-KOMA class
  \IfFileExists{parskip.sty}{%
    \usepackage{parskip}
  }{% else
    \setlength{\parindent}{0pt}
    \setlength{\parskip}{6pt plus 2pt minus 1pt}}
}{% if KOMA class
  \KOMAoptions{parskip=half}}
\makeatother
\usepackage{xcolor}
\usepackage[margin=1in]{geometry}
\usepackage{color}
\usepackage{fancyvrb}
\newcommand{\VerbBar}{|}
\newcommand{\VERB}{\Verb[commandchars=\\\{\}]}
\DefineVerbatimEnvironment{Highlighting}{Verbatim}{commandchars=\\\{\}}
% Add ',fontsize=\small' for more characters per line
\usepackage{framed}
\definecolor{shadecolor}{RGB}{248,248,248}
\newenvironment{Shaded}{\begin{snugshade}}{\end{snugshade}}
\newcommand{\AlertTok}[1]{\textcolor[rgb]{0.94,0.16,0.16}{#1}}
\newcommand{\AnnotationTok}[1]{\textcolor[rgb]{0.56,0.35,0.01}{\textbf{\textit{#1}}}}
\newcommand{\AttributeTok}[1]{\textcolor[rgb]{0.13,0.29,0.53}{#1}}
\newcommand{\BaseNTok}[1]{\textcolor[rgb]{0.00,0.00,0.81}{#1}}
\newcommand{\BuiltInTok}[1]{#1}
\newcommand{\CharTok}[1]{\textcolor[rgb]{0.31,0.60,0.02}{#1}}
\newcommand{\CommentTok}[1]{\textcolor[rgb]{0.56,0.35,0.01}{\textit{#1}}}
\newcommand{\CommentVarTok}[1]{\textcolor[rgb]{0.56,0.35,0.01}{\textbf{\textit{#1}}}}
\newcommand{\ConstantTok}[1]{\textcolor[rgb]{0.56,0.35,0.01}{#1}}
\newcommand{\ControlFlowTok}[1]{\textcolor[rgb]{0.13,0.29,0.53}{\textbf{#1}}}
\newcommand{\DataTypeTok}[1]{\textcolor[rgb]{0.13,0.29,0.53}{#1}}
\newcommand{\DecValTok}[1]{\textcolor[rgb]{0.00,0.00,0.81}{#1}}
\newcommand{\DocumentationTok}[1]{\textcolor[rgb]{0.56,0.35,0.01}{\textbf{\textit{#1}}}}
\newcommand{\ErrorTok}[1]{\textcolor[rgb]{0.64,0.00,0.00}{\textbf{#1}}}
\newcommand{\ExtensionTok}[1]{#1}
\newcommand{\FloatTok}[1]{\textcolor[rgb]{0.00,0.00,0.81}{#1}}
\newcommand{\FunctionTok}[1]{\textcolor[rgb]{0.13,0.29,0.53}{\textbf{#1}}}
\newcommand{\ImportTok}[1]{#1}
\newcommand{\InformationTok}[1]{\textcolor[rgb]{0.56,0.35,0.01}{\textbf{\textit{#1}}}}
\newcommand{\KeywordTok}[1]{\textcolor[rgb]{0.13,0.29,0.53}{\textbf{#1}}}
\newcommand{\NormalTok}[1]{#1}
\newcommand{\OperatorTok}[1]{\textcolor[rgb]{0.81,0.36,0.00}{\textbf{#1}}}
\newcommand{\OtherTok}[1]{\textcolor[rgb]{0.56,0.35,0.01}{#1}}
\newcommand{\PreprocessorTok}[1]{\textcolor[rgb]{0.56,0.35,0.01}{\textit{#1}}}
\newcommand{\RegionMarkerTok}[1]{#1}
\newcommand{\SpecialCharTok}[1]{\textcolor[rgb]{0.81,0.36,0.00}{\textbf{#1}}}
\newcommand{\SpecialStringTok}[1]{\textcolor[rgb]{0.31,0.60,0.02}{#1}}
\newcommand{\StringTok}[1]{\textcolor[rgb]{0.31,0.60,0.02}{#1}}
\newcommand{\VariableTok}[1]{\textcolor[rgb]{0.00,0.00,0.00}{#1}}
\newcommand{\VerbatimStringTok}[1]{\textcolor[rgb]{0.31,0.60,0.02}{#1}}
\newcommand{\WarningTok}[1]{\textcolor[rgb]{0.56,0.35,0.01}{\textbf{\textit{#1}}}}
\usepackage{graphicx}
\makeatletter
\def\maxwidth{\ifdim\Gin@nat@width>\linewidth\linewidth\else\Gin@nat@width\fi}
\def\maxheight{\ifdim\Gin@nat@height>\textheight\textheight\else\Gin@nat@height\fi}
\makeatother
% Scale images if necessary, so that they will not overflow the page
% margins by default, and it is still possible to overwrite the defaults
% using explicit options in \includegraphics[width, height, ...]{}
\setkeys{Gin}{width=\maxwidth,height=\maxheight,keepaspectratio}
% Set default figure placement to htbp
\makeatletter
\def\fps@figure{htbp}
\makeatother
\setlength{\emergencystretch}{3em} % prevent overfull lines
\providecommand{\tightlist}{%
  \setlength{\itemsep}{0pt}\setlength{\parskip}{0pt}}
\setcounter{secnumdepth}{-\maxdimen} % remove section numbering
\ifLuaTeX
  \usepackage{selnolig}  % disable illegal ligatures
\fi
\IfFileExists{bookmark.sty}{\usepackage{bookmark}}{\usepackage{hyperref}}
\IfFileExists{xurl.sty}{\usepackage{xurl}}{} % add URL line breaks if available
\urlstyle{same}
\hypersetup{
  pdftitle={IXIS Technical Exercise},
  pdfauthor={Will Smialek},
  hidelinks,
  pdfcreator={LaTeX via pandoc}}

\title{IXIS Technical Exercise}
\author{Will Smialek}
\date{2024-05-16}

\begin{document}
\maketitle

\hypertarget{setup}{%
\subsection{Setup}\label{setup}}

\begin{Shaded}
\begin{Highlighting}[]
\NormalTok{knitr}\SpecialCharTok{::}\NormalTok{opts\_chunk}\SpecialCharTok{$}\FunctionTok{set}\NormalTok{(}\AttributeTok{echo =} \ConstantTok{TRUE}\NormalTok{, }\AttributeTok{fig.align =} \StringTok{\textquotesingle{}center\textquotesingle{}}\NormalTok{)}

\CommentTok{\# Load libraries}
\NormalTok{pacman}\SpecialCharTok{::}\FunctionTok{p\_load}\NormalTok{(tidyverse, openxlsx, lubridate)}

\CommentTok{\# Read Google Analytics datasets}
\NormalTok{adds\_to\_cart }\OtherTok{\textless{}{-}} \FunctionTok{read.csv}\NormalTok{(}\StringTok{\textquotesingle{}DataAnalyst\_Ecom\_data\_addsToCart.csv\textquotesingle{}}\NormalTok{)}
\NormalTok{session\_counts }\OtherTok{\textless{}{-}} \FunctionTok{read.csv}\NormalTok{(}\StringTok{\textquotesingle{}DataAnalyst\_Ecom\_data\_sessionCounts.csv\textquotesingle{}}\NormalTok{)}
\end{Highlighting}
\end{Shaded}

\hypertarget{preprocessing}{%
\subsection{Preprocessing}\label{preprocessing}}

In order to create a data frame with month * device aggregation, the
first step is to break the \texttt{dim\_date} column into two parts
(year and month).

\begin{Shaded}
\begin{Highlighting}[]
\NormalTok{session\_counts }\OtherTok{\textless{}{-}}\NormalTok{ session\_counts }\SpecialCharTok{|\textgreater{}}
  
  \CommentTok{\# Change dim\_date type from character to Date}
  \FunctionTok{mutate}\NormalTok{(}\AttributeTok{dim\_date =} \FunctionTok{as.Date}\NormalTok{(dim\_date, }\AttributeTok{format =} \StringTok{\textquotesingle{}\%m/\%d/\%y\textquotesingle{}}\NormalTok{)) }\SpecialCharTok{|\textgreater{}}
  
  \CommentTok{\# Create columns for month and year}
  \FunctionTok{mutate}\NormalTok{(}\AttributeTok{dim\_year =} \FunctionTok{year}\NormalTok{(dim\_date),}
         \AttributeTok{dim\_month =} \FunctionTok{month}\NormalTok{(dim\_date))}

\CommentTok{\# Display preprocessed data}
\FunctionTok{tibble}\NormalTok{(session\_counts)}
\end{Highlighting}
\end{Shaded}

\begin{verbatim}
## # A tibble: 7,734 x 8
##    dim_browser       dim_deviceCategory dim_date   sessions transactions   QTY
##    <chr>             <chr>              <date>        <int>        <int> <int>
##  1 Safari            tablet             2012-07-01     2928          127   221
##  2 Internet Explorer desktop            2012-07-01     1106           28     0
##  3 Chrome            tablet             2012-07-01      474            3    13
##  4 Amazon Silk       tablet             2012-07-01      235            4     5
##  5 Internet Explorer mobile             2012-07-01      178            6    11
##  6 Internet Explorer tablet             2012-07-01      120            7     0
##  7 Android Browser   mobile             2012-07-01       10            0     0
##  8 error             desktop            2012-07-01        9            0     0
##  9 Edge              mobile             2012-07-01        5            0     0
## 10 Opera             mobile             2012-07-01        4            0     0
## # i 7,724 more rows
## # i 2 more variables: dim_year <dbl>, dim_month <dbl>
\end{verbatim}

\hypertarget{aggregation}{%
\subsection{Aggregation}\label{aggregation}}

Now that \texttt{session\_counts} has a year and month column, we can do
month * device aggregation. First, let's explore the data a bit more.

\begin{Shaded}
\begin{Highlighting}[]
\FunctionTok{table}\NormalTok{(session\_counts}\SpecialCharTok{$}\NormalTok{dim\_deviceCategory)}
\end{Highlighting}
\end{Shaded}

\begin{verbatim}
## 
## desktop  mobile  tablet 
##    2672    3013    2049
\end{verbatim}

This shows us that there are three categories of devices in our data
frame. This means each month should have at most three device categories
for each month. Now that we know what to expect, let's aggregate the
data accordingly.

\begin{Shaded}
\begin{Highlighting}[]
\CommentTok{\# Suppress friendly warning that appears when using \textasciigrave{}summarise\textasciigrave{} with more than one grouping}
\FunctionTok{options}\NormalTok{(}\AttributeTok{dplyr.summarise.inform =} \ConstantTok{FALSE}\NormalTok{)}

\NormalTok{month\_device\_aggregated\_data }\OtherTok{\textless{}{-}}\NormalTok{ session\_counts }\SpecialCharTok{|\textgreater{}}
  
  \CommentTok{\# Group by year, month, and device category}
  \FunctionTok{group\_by}\NormalTok{(dim\_year, dim\_month, dim\_deviceCategory) }\SpecialCharTok{|\textgreater{}}
  
  \CommentTok{\# For each group take sum of sessions, transactions, and QTY}
  \FunctionTok{summarise}\NormalTok{(}\AttributeTok{sessions =} \FunctionTok{sum}\NormalTok{(sessions),}
            \AttributeTok{transactions =} \FunctionTok{sum}\NormalTok{(transactions),}
            \AttributeTok{QTY =} \FunctionTok{sum}\NormalTok{(QTY)) }\SpecialCharTok{|\textgreater{}}
  
  \CommentTok{\# Calculate ECR for each group}
  \FunctionTok{mutate}\NormalTok{(}\AttributeTok{ECR =}\NormalTok{ transactions }\SpecialCharTok{/}\NormalTok{ sessions)}

\CommentTok{\# Display aggregated data}
\FunctionTok{tibble}\NormalTok{(month\_device\_aggregated\_data)}
\end{Highlighting}
\end{Shaded}

\begin{verbatim}
## # A tibble: 36 x 7
##    dim_year dim_month dim_deviceCategory sessions transactions   QTY     ECR
##       <dbl>     <dbl> <chr>                 <int>        <int> <int>   <dbl>
##  1     2012         7 desktop              335429        10701 18547 0.0319 
##  2     2012         7 mobile               274443         2576  4557 0.00939
##  3     2012         7 tablet               158717         4884  8700 0.0308 
##  4     2012         8 desktop              392079        12912 23316 0.0329 
##  5     2012         8 mobile               275556         3165  5572 0.0115 
##  6     2012         8 tablet               154858         3202  5760 0.0207 
##  7     2012         9 desktop              272771         8898 16507 0.0326 
##  8     2012         9 mobile               220689         2381  4050 0.0108 
##  9     2012         9 tablet               169193         4379  7869 0.0259 
## 10     2012        10 desktop              302682         9373 17675 0.0310 
## # i 26 more rows
\end{verbatim}

Here we have 36 rows which contains the month * device aggregated data.

\hypertarget{month-over-month-comparison}{%
\subsection{Month Over Month
Comparison}\label{month-over-month-comparison}}

In order to compare the most recent two months, we must first determine
which rows in the \texttt{session\_counts} are the two most recent.
Visually, it looks as if \texttt{session\_counts} has been given to us
sorted by date, ascending. However, we should write code that does not
assume that the data has already been sorted.

\begin{Shaded}
\begin{Highlighting}[]
\FunctionTok{groups}\NormalTok{(month\_device\_aggregated\_data)}
\end{Highlighting}
\end{Shaded}

\begin{verbatim}
## [[1]]
## dim_year
## 
## [[2]]
## dim_month
\end{verbatim}

This shows us that \texttt{month\_device\_aggregated\_data} currently
has two groups, \texttt{dim\_year} and \texttt{dim\_month}. These are
the exact groups we would like in order to summarize data by month. We
will be able to include the \texttt{addsToCart} column from the second
csv file as soon as we aggregate by month.

\begin{Shaded}
\begin{Highlighting}[]
\NormalTok{recent\_two\_months }\OtherTok{\textless{}{-}}\NormalTok{ month\_device\_aggregated\_data }\SpecialCharTok{|\textgreater{}}
  
  \CommentTok{\# Sum relevant data by month}
  \FunctionTok{summarise}\NormalTok{(}\AttributeTok{sessions =} \FunctionTok{sum}\NormalTok{(sessions),}
            \AttributeTok{transactions=} \FunctionTok{sum}\NormalTok{(transactions),}
            \AttributeTok{QTY =} \FunctionTok{sum}\NormalTok{(QTY)) }\SpecialCharTok{|\textgreater{}}
  
  \CommentTok{\# Calculate ECR for each month}
  \FunctionTok{mutate}\NormalTok{(}\AttributeTok{ECR =}\NormalTok{ transactions }\SpecialCharTok{/}\NormalTok{ sessions) }\SpecialCharTok{|\textgreater{}}
  
  \CommentTok{\# Perform a left join to include \textasciigrave{}addsToCart\textasciigrave{}}
  \FunctionTok{left\_join}\NormalTok{(adds\_to\_cart, }\AttributeTok{by =} \FunctionTok{c}\NormalTok{(}\StringTok{\textquotesingle{}dim\_year\textquotesingle{}}\NormalTok{, }\StringTok{\textquotesingle{}dim\_month\textquotesingle{}}\NormalTok{)) }\SpecialCharTok{|\textgreater{}}
  
  \CommentTok{\# Sort rows to have most recent months at the top}
  \FunctionTok{arrange}\NormalTok{(}\FunctionTok{desc}\NormalTok{(dim\_year), }\FunctionTok{desc}\NormalTok{(dim\_month)) }\SpecialCharTok{|\textgreater{}}
  
  \CommentTok{\# After \textasciigrave{}summarise\textasciigrave{}, there is one grouping left}
  \CommentTok{\# This grouping must be removed in order to slice properly}
  \FunctionTok{ungroup}\NormalTok{() }\SpecialCharTok{|\textgreater{}}
  
  \CommentTok{\# Slice the top two rows}
  \FunctionTok{slice}\NormalTok{(}\DecValTok{1}\SpecialCharTok{:}\DecValTok{2}\NormalTok{)}

\CommentTok{\# Display current data}
\FunctionTok{tibble}\NormalTok{(recent\_two\_months)}
\end{Highlighting}
\end{Shaded}

\begin{verbatim}
## # A tibble: 2 x 7
##   dim_year dim_month sessions transactions   QTY    ECR addsToCart
##      <dbl>     <dbl>    <int>        <int> <int>  <dbl>      <int>
## 1     2013         6  1388834        34538 61891 0.0249     107970
## 2     2013         5  1164639        28389 51629 0.0244     136720
\end{verbatim}

This shows we've successfully grabbed the most recent two months. Before
we calculate direct comparisons between these two months, let's reorder
them chronologically.

\begin{Shaded}
\begin{Highlighting}[]
\NormalTok{recent\_two\_months }\OtherTok{\textless{}{-}}\NormalTok{ recent\_two\_months }\SpecialCharTok{|\textgreater{}}
  
  \CommentTok{\# Rearrange rows to be chronological}
  \FunctionTok{arrange}\NormalTok{(dim\_year, dim\_month) }\SpecialCharTok{|\textgreater{}}

  \CommentTok{\# Create description column. This column will simply have which month it is}
  \CommentTok{\# for our current two rows, but it will describe future rows differently}
  \FunctionTok{mutate}\NormalTok{(}\AttributeTok{description =} \FunctionTok{paste}\NormalTok{(dim\_month, dim\_year, }\AttributeTok{sep =} \StringTok{\textquotesingle{}/\textquotesingle{}}\NormalTok{)) }\SpecialCharTok{|\textgreater{}}

  \CommentTok{\# Reorder columns so that the description is first and}
  \CommentTok{\# remove now{-}redundant date columns}
  \FunctionTok{select}\NormalTok{(description, }\FunctionTok{everything}\NormalTok{(), }\SpecialCharTok{{-}}\NormalTok{dim\_year, }\SpecialCharTok{{-}}\NormalTok{dim\_month)}

\CommentTok{\# Display current data}
\FunctionTok{tibble}\NormalTok{(recent\_two\_months)}
\end{Highlighting}
\end{Shaded}

\begin{verbatim}
## # A tibble: 2 x 6
##   description sessions transactions   QTY    ECR addsToCart
##   <chr>          <int>        <int> <int>  <dbl>      <int>
## 1 5/2013       1164639        28389 51629 0.0244     136720
## 2 6/2013       1388834        34538 61891 0.0249     107970
\end{verbatim}

We've combined the month and year columns for these two rows to serve as
their description. Now we will be able to create new rows which describe
diferent comparisons - absolute difference and relative difference.
Absolute difference between \(a\) and \(b\) is written \(|a - b|\).
Relative difference between \(a\) and \(b\) is calculated as
\(\frac{b-a}{b}\).

\begin{Shaded}
\begin{Highlighting}[]
\CommentTok{\# Create absolute difference row}
\NormalTok{absolute\_difference }\OtherTok{\textless{}{-}}\NormalTok{ recent\_two\_months }\SpecialCharTok{|\textgreater{}}
  \FunctionTok{summarise}\NormalTok{(}
    \AttributeTok{description =} \StringTok{\textquotesingle{}absolute difference\textquotesingle{}}\NormalTok{,}
    
    \CommentTok{\# Calculate absolute difference for each numeric column}
    \FunctionTok{across}\NormalTok{(}\FunctionTok{where}\NormalTok{(is.numeric), }\SpecialCharTok{\textasciitilde{}} \FunctionTok{abs}\NormalTok{(.[}\DecValTok{1}\NormalTok{] }\SpecialCharTok{{-}}\NormalTok{ .[}\DecValTok{2}\NormalTok{]))}
\NormalTok{  )}

\CommentTok{\# Create relative difference row}
\NormalTok{relative\_difference }\OtherTok{\textless{}{-}}\NormalTok{ recent\_two\_months }\SpecialCharTok{|\textgreater{}}
  \FunctionTok{summarise}\NormalTok{(}
    \AttributeTok{description =} \StringTok{\textquotesingle{}relative difference\textquotesingle{}}\NormalTok{,}
    
    \CommentTok{\# Calculate relative difference for each numeric column}
    \FunctionTok{across}\NormalTok{(}\FunctionTok{where}\NormalTok{(is.numeric), }\SpecialCharTok{\textasciitilde{}}\NormalTok{ (.[}\DecValTok{2}\NormalTok{] }\SpecialCharTok{{-}}\NormalTok{ .[}\DecValTok{1}\NormalTok{]) }\SpecialCharTok{/}\NormalTok{ .[}\DecValTok{1}\NormalTok{])}
\NormalTok{  )}

\CommentTok{\# Bind new rows to bottom of \textasciigrave{}recent\_two\_months\textasciigrave{}}
\NormalTok{month\_over\_month\_comp }\OtherTok{\textless{}{-}} \FunctionTok{rbind}\NormalTok{(recent\_two\_months, absolute\_difference, relative\_difference)}

\CommentTok{\# Display month over month comparison}
\FunctionTok{tibble}\NormalTok{(month\_over\_month\_comp)}
\end{Highlighting}
\end{Shaded}

\begin{verbatim}
## # A tibble: 4 x 6
##   description            sessions transactions       QTY      ECR addsToCart
##   <chr>                     <dbl>        <dbl>     <dbl>    <dbl>      <dbl>
## 1 5/2013              1164639        28389     51629     0.0244   136720    
## 2 6/2013              1388834        34538     61891     0.0249   107970    
## 3 absolute difference  224195         6149     10262     0.000493  28750    
## 4 relative difference       0.193        0.217     0.199 0.0202       -0.210
\end{verbatim}

This shows the absolute and relative difference between the two most
recent months. Note that while absolute difference has the same units as
the first two rows, relative difference does not. The units of relative
difference can be described as the \emph{proportional change} between
the two most recent months.

\hypertarget{create-spreadsheet}{%
\subsection{Create spreadsheet}\label{create-spreadsheet}}

Both output data frames have been created,
\texttt{month\_device\_aggregated\_data} and
\texttt{month\_over\_month\_comp}. All that's left to do is to create
the Excel spreadsheet to store them.

\begin{Shaded}
\begin{Highlighting}[]
\NormalTok{wb }\OtherTok{\textless{}{-}} \FunctionTok{createWorkbook}\NormalTok{()}

\CommentTok{\# Add first sheet}
\FunctionTok{addWorksheet}\NormalTok{(wb, }\StringTok{\textquotesingle{}Aggregated Data\textquotesingle{}}\NormalTok{)}
\FunctionTok{writeData}\NormalTok{(wb, }\StringTok{\textquotesingle{}Aggregated Data\textquotesingle{}}\NormalTok{, month\_device\_aggregated\_data)}

\CommentTok{\# Add second sheet}
\FunctionTok{addWorksheet}\NormalTok{(wb, }\StringTok{\textquotesingle{}Month Over Month Comparison\textquotesingle{}}\NormalTok{)}
\FunctionTok{writeData}\NormalTok{(wb, }\StringTok{\textquotesingle{}Month Over Month Comparison\textquotesingle{}}\NormalTok{, month\_over\_month\_comp)}

\CommentTok{\# Add percentage style to apply to the relative difference row in the}
\CommentTok{\# month over month comparison (for readability)}
\FunctionTok{addStyle}\NormalTok{(wb,}
         \AttributeTok{sheet =} \StringTok{\textquotesingle{}Month Over Month Comparison\textquotesingle{}}\NormalTok{,}
         \AttributeTok{rows =} \DecValTok{5}\NormalTok{, }\CommentTok{\# 4th row (+1 to account for header)}
         \AttributeTok{cols =} \DecValTok{1}\SpecialCharTok{:}\FunctionTok{ncol}\NormalTok{(month\_over\_month\_comp),}
         \AttributeTok{style =} \FunctionTok{createStyle}\NormalTok{(}\AttributeTok{numFmt =} \StringTok{"0.00\%"}\NormalTok{))}

\CommentTok{\# For both sheets, set all column widths to be auto{-}calculated (for readability)}
\FunctionTok{setColWidths}\NormalTok{(wb, }\AttributeTok{sheet =} \StringTok{\textquotesingle{}Aggregated Data\textquotesingle{}}\NormalTok{, }\AttributeTok{cols =} \DecValTok{1}\SpecialCharTok{:}\FunctionTok{ncol}\NormalTok{(month\_device\_aggregated\_data), }\AttributeTok{widths =} \StringTok{"auto"}\NormalTok{)}
\FunctionTok{setColWidths}\NormalTok{(wb, }\AttributeTok{sheet =} \StringTok{\textquotesingle{}Month Over Month Comparison\textquotesingle{}}\NormalTok{, }\AttributeTok{cols =} \DecValTok{1}\SpecialCharTok{:}\FunctionTok{ncol}\NormalTok{(month\_over\_month\_comp), }\AttributeTok{widths =} \StringTok{"auto"}\NormalTok{)}

\CommentTok{\# Write spreadsheet to \textasciigrave{}output.xlsx\textasciigrave{}}
\FunctionTok{saveWorkbook}\NormalTok{(wb, }\StringTok{\textquotesingle{}output.xlsx\textquotesingle{}}\NormalTok{, }\AttributeTok{overwrite =} \ConstantTok{TRUE}\NormalTok{)}
\end{Highlighting}
\end{Shaded}


\end{document}
